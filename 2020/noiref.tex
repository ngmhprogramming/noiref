\documentclass{article}
\usepackage[utf8]{inputenc}
\usepackage[a4paper, left=1in, right=1in, top=1in, bottom=1in]{geometry}
\usepackage{listings}
\usepackage{xcolor}
\usepackage{float}

\lstset{
    frame=tb,
    tabsize=4,
    showstringspaces=false,
    numbers=left,
    commentstyle=\color{purple},
    keywordstyle=\color{orange},
    numberstyle=\color{gray},
    stringstyle=\color{green},
    basicstyle=\ttfamily
}

\title{noiref}
\author{ngmh}
\date{December 2019}

\begin{document}

\maketitle
\tableofcontents
\newpage

\section{Data Structures}

\begin{table}[H]
\begin{tabular}{|c|c|c|c|c|}
\hline
Data Structure & Precomputation / Update & Query    & Memory     & Notes                                \\ \hline
Prefix Sum     & O(N) / X                & O(1)     & O(N)       & Associative Functions (+, XOR)       \\ \hline
Sparse Table   & O(N log N) / X          & O(1)     & O(N log N) & Non-Associative Functions (max, gcd) \\ \hline
Fenwick Tree   & X / O(log N)            & O(log N) & O(N)       & Prefix Sum with Updates              \\ \hline
Segment Tree   & X / O(log N)            & O(log N) & O(4N)      & Allows more Information              \\ \hline
\end{tabular}
\caption{}
\label{tab:my-table}
\end{table}

\subsection{Prefix Sums}

\begin{flushleft}
Prefix Sums rely on the Principle of Inclusion and Exclusion.
By adding and subtracting the correct prefixes, we can determine
the answer for any subarray.
\newline
This idea can be extended to multiple dimensions as well, but beyond 2
it gets slightly cancerous.
\newline
Query: O(1)
\newline
Update: X
\end{flushleft}

\subsubsection{1D}
\lstinputlisting[language=C++]{ds_ps_1d.cpp}
\subsubsection{2D}
\lstinputlisting[language=C++]{ds_ps_2d.cpp}

\subsection{Sparse Table}

\begin{flushleft}
Sparse Tables to me, are Prefix Sums on steroids.
Instead of using prefixes, we use subarrays with sizes which are powers of 2.
Then, any query will need at most 2 of the calculated subarrays.
\newline
Query: O(1)
\newline
Update: X
\end{flushleft}

\subsubsection{1D}
\lstinputlisting[language=C++]{ds_sp_1d.cpp}
\subsubsection{2D}
\lstinputlisting[language=C++]{ds_sp_2d.cpp}

\subsection{Fenwick Trees}

\begin{flushleft}
Fenwick Trees essentially act the same as prefix sums,
except that they can perform updates. However, the complexity of code
varies depending on what kind of queries and updates are needed.
One cool use case is for range maximum, but only works when updates are strictly
non-decreasing.
\newline
Query: O(log N)
\newline
Update: O(log N)
\end{flushleft}

\subsubsection{Point Update, Range Query}
\lstinputlisting[language=C++]{ds_fw_purq.cpp}
\subsubsection{Range Update, Point Query}
\lstinputlisting[language=C++]{ds_fw_rupq.cpp}
\subsubsection{Range Update, Range Query}
\lstinputlisting[language=C++]{ds_fw_rurq.cpp}

\subsection{Segment Trees}

\begin{flushleft}
Segment Trees need more memory and code to implement, but are much more powerful,
since more information can be stored in each node. 
They can be used for anything really, e.g. finding the Kth largest element, and Maxsum.
\newline
Query: O(log N)
\newline
Update: O(log N)
\end{flushleft}

\subsubsection{1D}
\lstinputlisting[language=C++]{ds_st_1d.cpp}
\subsubsection{Lazy Propagation - Range Add}
\lstinputlisting[language=C++]{ds_st_lazy.cpp}
\subsubsection{Lazy Propagation - Range Add + Set}
\lstinputlisting[language=C++]{ds_st_lazy_rs.cpp}
\subsubsection{Maxsum Tree}
\lstinputlisting[language=C++]{ds_st_ms.cpp}
\subsubsection{Order Statistic Tree - More than K}
\lstinputlisting[language=C++]{ds_st_os.cpp}
\subsubsection{Order Statistic Tree - More than K + Updates}
\lstinputlisting[language=C++]{ds_st_osu.cpp}
\subsubsection{2D}
\lstinputlisting[language=C++]{ds_st_2d.cpp}

\section{Graph Theory}


\begin{table}[H]
\begin{tabular}{|c|c|c|}
\hline
Graph Algorithm     & Complexity              & Notes                       \\ \hline
DFS                 & O(N)                    & Connectedness               \\ \hline
BFS                 & O(N)                    & Unweighted Shortest Path    \\ \hline
Floyd-Warshall      & O(N\textasciicircum{}3) & All Pairs Shortest Path     \\ \hline
Dijkstra            & O(E log E)              & Single Source Shortest Path \\ \hline
TSP                 & O(2\textasciicircum{}N) & Visiting All Nodes          \\ \hline
UFDS                & O(1)                    & Amortized                   \\ \hline
MST - Kruskal       & O(E)                    & Greedy                      \\ \hline
MST - Prim's        & O(E log E)              & Dijkstra                    \\ \hline
Bipartite Matching  & ?                       & 2 Sets of Nodes             \\ \hline
Articulation Points & ?                       & Node Splits Graph           \\ \hline
Bridges             & ?                       & Edge Splits Graph           \\ \hline
SCC                 & ?                       & Cycles                      \\ \hline
\end{tabular}
\caption{Quick Summary of General Graph Algorithms}
\label{tab:gt}
\end{table}

\subsection{Depth First Search}
\lstinputlisting[language=C++]{graph_dfs.cpp}
\subsection{Breadth First Search}
\lstinputlisting[language=C++]{graph_bfs.cpp}
\subsection{Floyd-Warshall}
\lstinputlisting[language=C++]{graph_floyd.cpp}
\subsection{Dijkstra}
\lstinputlisting[language=C++]{graph_dijkstra.cpp}
\subsection{Travelling Salesman Problem}
\lstinputlisting[language=C++]{graph_tsp.cpp}
\subsection{Union Find Disjoint Subset}
\lstinputlisting[language=C++]{graph_ufds.cpp}

\subsection{Minimum Spanning Tree}
\subsubsection{Kruskal}
\lstinputlisting[language=C++]{graph_mst_kruskal.cpp}
\subsubsection{Prim's}
\lstinputlisting[language=C++]{graph_mst_prims.cpp}

\subsection{Bipartite Matching}
\lstinputlisting[language=C++]{graph_bm.cpp}
\subsection{Articulation Points}
\lstinputlisting[language=C++]{graph_atp.cpp}
\subsection{Bridges}
\lstinputlisting[language=C++]{graph_bridges.cpp}
\subsection{Strongly Connected Components}
\lstinputlisting[language=C++]{graph_scc.cpp}

\subsection{Trees}
\subsubsection{Diameter}
\lstinputlisting[language=C++]{graph_tree_diam.cpp}
\subsubsection{$2^{K}$ Decomposition}
\lstinputlisting[language=C++]{graph_tree_2k.cpp}
\subsubsection{Lowest Common Ancestor}
\lstinputlisting[language=C++]{graph_tree_lca.cpp}
\subsubsection{All Pairs Shortest Path}
\lstinputlisting[language=C++]{graph_tree_apsp.cpp}
\subsubsection{Preorder}
\lstinputlisting[language=C++]{graph_tree_pre.cpp}
\subsubsection{Postorder}
\lstinputlisting[language=C++]{graph_tree_post.cpp}
\subsubsection{Subtree to Range}
\lstinputlisting[language=C++]{graph_tree_range.cpp}
\subsubsection{Leaf Pruning}
\lstinputlisting[language=C++]{graph_tree_leaf.cpp}
\subsubsection{Weighted Maximum Independent Set}
\lstinputlisting[language=C++]{graph_tree_wmis.cpp}
\subsubsection{Heavy-Light Decomposition}
\lstinputlisting[language=C++]{graph_tree_hld.cpp}
\subsubsection{Centroid Decomoposition}
\lstinputlisting[language=C++]{graph_tree_cd.cpp}

\section{Dynamic Programming}

\begin{table}[H]
\begin{tabular}{|c|c|c|}
\hline
DP Algorithm        & Complexity              & Notes     \\ \hline
Coin Change         & O(NV)                   &           \\ \hline
Coin Combinations   & O(NV)                   &           \\ \hline
Knapsack - 0-1      & O(NS)                   &           \\ \hline
Knapsack - 0-K      & O(log2(K)+NS)           &           \\ \hline
LIS - Naive         & O(N\textasciicircum{}2) &           \\ \hline
LIS - DS / Optimal  & O(N log N)              &           \\ \hline
LCS                 & O(N\textasciicircum{}2) &           \\ \hline
LCS - LIS           & O(N log N)              &           \\ \hline
Digits              & O(10|N|)                &           \\ \hline
Convex Hull Speedup & O(N)                    & Amortized \\ \hline
Divide and Conquer  & O(N log N)              &           \\ \hline
\end{tabular}
\caption{Quick Summary of General Dynamic Programming Algorithms}
\label{tab:dp}
\end{table}

\subsection{Coin Change}
\lstinputlisting[language=C++]{dp_coin_change.cpp}
\subsection{Coin Combinations}
\lstinputlisting[language=C++]{dp_coin_combs.cpp}

\subsection{Knapsack}
\subsubsection{0-1}
\lstinputlisting[language=C++]{dp_knap_01.cpp}
\subsubsection{0-K}
\lstinputlisting[language=C++]{dp_knap_0k.cpp}

\subsection{Longest Increasing Subsequence}
\subsubsection{$N^{2}$}
\lstinputlisting[language=C++]{dp_lis_2.cpp}
\subsubsection{$N log N$}
\lstinputlisting[language=C++]{dp_lis_log_fw.cpp}
\subsubsection{Optimal}
\lstinputlisting[language=C++]{dp_lis_log.cpp}

\subsection{Longest Common Subsequence}
\subsubsection{$N^{2}$}
\lstinputlisting[language=C++]{dp_lcs_2.cpp}
\subsubsection{Longest Increasing Subsequence}
\lstinputlisting[language=C++]{dp_lcs_lis.cpp}

\subsection{Digits}
\lstinputlisting[language=C++]{dp_digits.cpp}

\subsection{Convex Hull Speedup}

\begin{flushleft}
This speeds up any DP which is a quadratic function.
A similar idea can also be done for linear functions but just using a set.

The important part is knowing how to rearrange the transition to get coefficients.

Query: O(1)
Update: O(1)
\end{flushleft}


\lstinputlisting[language=C++]{dp_ch.cpp}
\subsection{Divide and Conquer}
\lstinputlisting[language=C++]{dp_dnc.cpp}

\section{Math}
\subsection{Greatest Common Divisor}
\lstinputlisting[language=C++]{math_gcd.cpp}
\subsection{Lowest Common Multiple}
\lstinputlisting[language=C++]{math_lcm.cpp}

\subsection{Modular Functions}
\subsubsection{Multiplication}
\lstinputlisting[language=C++]{math_mulmod.cpp}
\subsubsection{Exponentiation}
\lstinputlisting[language=C++]{math_powmod.cpp}

\subsection{Primes}
\subsubsection{Sieve of Eratosthenes}
\lstinputlisting[language=C++]{math_prime_sieve.cpp}
\subsubsection{Prime Factorisation}
\lstinputlisting[language=C++]{math_prime_fac.cpp}

\subsection{Fibonacci}
\lstinputlisting[language=C++]{math_fibo.cpp}
\subsection{$^{n}C_{k}$}
\lstinputlisting[language=C++]{math_nck.cpp}

\section{Algorithms}
\subsection{Discretisation}
\lstinputlisting[language=C++]{algo_ds.cpp}
\subsection{Binary Search}
\lstinputlisting[language=C++]{algo_bs.cpp}
\subsection{Mo's Algorithm}
\lstinputlisting[language=C++]{algo_mos.cpp}

\section{Miscellaneous}
\subsection{Macros + Functions + Variables}
\lstinputlisting[language=C++]{misc_mfv.cpp}

\subsection{Compile Commands}
\subsubsection{Compile}
\lstinputlisting[language=C++]{misc_cmp_compile.cpp}
\subsubsection{Build}
\lstinputlisting[language=C++]{misc_cmp_build.cpp}
\subsubsection{Command Line}
\lstinputlisting[language=C++]{misc_cmp_cmd.cpp}
\subsubsection{Simple Script}
\lstinputlisting[language=C++]{misc_cmp_script.cpp}

\subsection{Input/Output}
\subsubsection{Fast}
\lstinputlisting[language=C++]{misc_io_fast.cpp}
\subsubsection{Faster}
\lstinputlisting[language=C++]{misc_io_faster.cpp}

\subsection{Pruning}
\lstinputlisting[language=C++]{misc_prune.cpp}

\subsection{Optimise}
\lstinputlisting[language=C++]{misc_optimise.cpp}

\end{document}
